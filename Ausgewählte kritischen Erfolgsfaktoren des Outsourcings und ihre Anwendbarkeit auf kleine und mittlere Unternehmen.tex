\documentclass[seminar,colorlinks]{isih}

% verpflichtende Parameter:
%   "foseminar" / "beleg" / "bachelor" / "master" / "diplom" : Art der Abschlussarbeit
%   "seminar" wird ebenfalls unterstützt, es müssen dann allerdings alle Teilnehmer mit Namen und Matr.-Nummer je Komma-separiert in den "author"-Parameter geschrieben werden
% 
% optionale Parameter:
%   "colorlinks"    : farbige (statt schwarze) Links im Dokument
%   "singlespace"   : einfacher (statt größerer) Zeilenabstand
%   "indent"        : eingerückte Zeile bei neuem Absatz hinzufügen
%   "nopar"         : Leerzeile nach Absatz entfernen
%   "compmodern"    : Schriftart 'Computer Modern' statt 'Times New Roman'

%---------------------------------------------------------------------
% Platz für weitere Pakete
% \usepackage{...}
\usepackage[nameinlink]{cleveref}
\crefname{figure}{Abb.}{Abb.}
%---------------------------------------------------------------------

\begin{document}

%---------------------------------------------------------------------
% Settings
\title{Ausgewählte kritischen Erfolgsfaktoren des Outsourcings und ihre Anwendbarkeit auf kleine und mittlere Unternehmen} % Titel der Arbeit
\author{Haustein, Henry; Hettmann, Sebastian}                      % Name des Bearbeiters
\address{Königsteinstraße 8, 01277 Dresden; ...}    % Adresse des Bearbeiters
\matr{4685025, ...}                     % Matrikelnummer
\startdate{28.11.2020}              % Start-Datum der Arbeit
\submitdate{01.03.2021}             % Abgabe-Datum der Arbeit
\supervisor{Strahringer, Susanne}          % Namen aller Betreuer
\semester{WS 2020/21}               % aktuelles Semester
%---------------------------------------------------------------------

\isihtitlepage
\begin{preface}

% Abstract / Zusammenfassung
%\begin{abstract}
%    Dieses Dokument ist als Template für Abschlussarbeiten am ISIH-Lehrstuhl gedacht. Es beinhaltet alle notwendigen Informationen und Styles, wie sie üblicherweise angewandt werden. Allerdings sind die Vorgaben nicht bindend, d.h., sollten mit dem Betreuer abgesprochen werden.
%\end{abstract}

\tableofcontents
%\listoffigures
%\listoftables
%\lstlistoflistings\addcontentsline{toc}{section}{Quellcodeverzeichnis}  % entfernen, falls nicht benötigt
%\listofusedequations\addcontentsline{toc}{section}{Formelverzeichnis}   % entfernen, falls nicht benötigt

% Abkürzungsverzeichnis
%\listofabbreviations
%    \abbrev{TUD}{Technische Universität Dresden}
%    \abbrev{...}{...}

\end{preface}
%%%%%%%%%%%%%%%%%%%%%%%%%%%%%%%%%%%%%%%%%%%%%%%%%%%%%%%%%%%%%%%%%%%%%%%%
%   Beginn des Hauptteils

% Es bietet sich an, Kapitel in separate .tex-Dateien zu schreiben
% und diese dann per "\input" oder "\include" einzubinden, anstatt 
% alles in diese Haupt-Datei direkt rein zu schreiben. Gleiches 
% gilt für die Anhänge. 
% Aus technischen Gründen sind Kapitel als "\section" zu setzen.

\section{Einleitung}
Der globale Outsourcing-Markt wächst beständig. Das Marktforschungsinstitut ISG ermittelt jedes Quartal das kumulierte Volumen aller Outsourcing-Abschlüsse der Privatwirtschaft und öffentlichen Verwaltung von mindestens 4 Millionen Euro. Das jährliche Volumen der Abschlüsse der Region EMEA (Europa, naher Osten und Afrika) stieg dabei von 2014 bis 2019 von 12 Milliarden auf 19,2 Milliarden US-Dollar - ein Anstieg von 60 \% (siehe \cref{auftragsvolumen}).

\begin{figure}[h]
	\centering
	\includegraphics[width=0.75\textwidth]{./pics/testbild.jpg}
	\caption[Auftragsvolumen von Outsourcing-Verträgen]{Jährliches Auftragsvolumen von Outsourcing-Verträgen in Milliarden US-Dollar, Daten aus (Jain \& Sachs, 2020; King, 2017; Matthews, 2019), eigene Berechnungen}
	\label{auftragsvolumen}
\end{figure}

Das Outsourcing verspricht unter anderem Kostenreduktion und -transparenz, sowie einfachere Skalierbarkeit von Leistungen, aber verbunden mit einem hohen Risko: Hat man als Unternehmen erstmals einen Anbieter gefunden, so sind mit einem Anbieterwechsel hohe Kosten und großer Zeitaufwand verbunden (Ortner, 2015). Es kommt also auf die richtige Wahl des Outsourcing-Anbieters an. In den letzten 20 Jahren wurde daher viel Forschung betrieben, um die kritischen Erfolgsfaktoren des Outsourcings zu ermitteln und so so das Risiko einer schlechten Outsourcing-Entscheidung zu minimieren.

Die Ergebnisse dieser Forschung stammen aber häufig aus Untersuchungen von Outsourcing-Verträgen großer Firmen (Barthélemy, 2003), obwohl insbesondere Unternehmen mit wenigen Mitarbeitern, also kleine und mittlere Unternehmen, auf Outsourcing von zum Beispiel IT oder des Rechnungswesens angewiesen sind und dies auch tun (Embleton \& Wright, 1998) (siehe \cref{mitarbeiter}).

\begin{figure}[h]
	\centering
	\includegraphics[width=0.75\textwidth]{./pics/testbild.jpg}
	\caption[Mitarbeiteranzahl von Unternehmen, die Outsourcing nutzen]{Mitarbeiteranzahl von Unternehmen, die Outsourcing nutzen, aus (Embleton \& Wright, 1998)}
	\label{mitarbeiter}
\end{figure}

Es stellt sich also folgende Forschungsfrage: \textit{Sind die in der Forschung ermittelten kritischen Erfolgsfaktoren für Outsourcing auch für kleine und mittlere Unternehmen gültig?} Dazu sollen anhand einer Fallstudie die Outsourcing-Beziehungen eines Unternehmens auf Erfolg oder Misserfolg untersucht werden. Dabei soll überprüft werden, ob sich deren Ergebnis (Erfolg oder Misserfolg) auf die in einer Literaturrecherche ermittelten Erfolgsfaktoren zurückführen lässt. Auf diese Weise leistet die Arbeit einen Beitrag zur Anwendbarkeit von Erfolgsfaktoren, die für Großunternehmen ermittelt wurde, auf kleine und mittlere Unternehmen.

\section{Theoretisches Framework}
\subsection{Definition von Outsourcing}

Outsourcing beschreibt die Auslagerung von Unternehmensprozessen in andere Unternehmen. Diese Prozesse könnten auch unternehmensintern vorhanden sein, aber durch effektivere und effizientere Bearbeitung des Prozesses durch ein anderes Unternehmen kann ein Wettbewerbsvorteil entstehen. Es geht dabei nicht nur um das Kaufen von Produkten bzw. Dienstleistungen, sondern auch um das Abgeben von Verantwortung und Know-how an den Zulieferer (McCarthy \& Anagnostou, 2004).

Wie in \cref{dimensionen} dargestellt, gibt es verschiedene Dimensionen des Outsourcings: Unternehmenszugehörigkeit, Leistungsumfang, Zeit, Anzahl der Leistungsersteller und Entfernung zwischen den Outsourcing-Partnern.

\begin{figure}[h]
	\centering
	\includegraphics[width=0.75\textwidth]{./pics/testbild.jpg}
	\caption[Dimensionen des Outsourcings]{Dimensionen des Outsourcings, (Hermes \& Schwarz, 2005)}
	\label{dimensionen}
\end{figure}

Die Dimension Unternehmenszugehörigkeit hat die Ausprägungen externes und internes Outsourcing. Externes Outsourcing meint Outsourcing an Fremdfirmen außerhalb eines Konzerns; im Gegensatz dazu kann internes Outsourcing zum Beispiel das Outsourcing an eine andere Abteilung oder an eine Ausgründung sein. Beim internen Outsourcing macht sich ein Unternehmen nicht von Fremdfirmen abhängig und behält sein Know-how (Hermes \& Schwarz, 2005).

Das Auslagern ganzer Prozesse wie zum Beispiel Buchführung, Logistik oder Einkauf wird als Business Process Outsourcing bezeichnet und ist insbesondere dann sinnvoll, wenn sich einzelne Funktionen des Unternehmens nicht leicht isolieren lassen (Dittrich \& Braun, 2004). Beim partiellen Outsourcing werden nur einzelne Aufgaben, zum Beispiel Übersetzung von Texten, an den Outsourcing-Partner übergeben. Der Outsourcing-Partner kennt den unternehmensinternen Zusammenhang aber nicht. Wenn sich Prozesse im Unternehmen gut abgrenzen lassen, wie zum Beispiel ein Sicherheitsdienst, kann ein totales Outsourcing sinnvoll sein. Die Anforderungen lassen sich hierbei gut definieren und messen (Anderson, 1997).

Wird eine Outsourcing-Entscheidung wieder rückgängig gemacht, spricht man von Backsourcing. IT-Dienstleistungen werden häufig aufgrund ihrer hohen Kosten an Fremdfirmen abgegeben, aber wegen schlechter Qualität wieder ins eigene Unternehmen zurückgeholt (von Bary et al., 2020). Beim Insourcing evaluiert man die Outsourcing-Option, versucht aber das gleiche Ergebnis des Outsourcings mit internen Mitteln zu erreichen (Hirschheim \& Lacity, 2000).

Gibt es für ein Thema oder Aufgabenbereich nur einen Outsourcing-Partner, so handelt es sich um Single-Sourcing; bei mehreren Partnern um Multi-Sourcing. Beim Single-Sourcing läuft man als Unternehmen Gefahr, dass man sich in einen zu große Abhängigkeit des Partners begibt. Multi-Sourcing löst dieses Problem, hat aber einen Preis: mehrere Transaktionen erhöhen die Transaktionskosten. Zudem ist ein gutes Management der Outsourcing-Beziehungen notwendig (Hermes \& Schwarz, 2005).

Beim Offshoring sitzt der Outsourcing-Partner in Ländern mit deutlich geringen Lohnkosten; häufigstes Beispiel ist hierfür Indien (Bongartz, 2003; Farrell, 2004). Beim Nearshoring sitzt der Outsourcing-Partner geografisch näher am eigenen Unternehmen, das heißt es werden nur Unternehmens- und Landesgrenzen überquert, aber keine Zeitzonen (Zieris \& Salinger, 2013), Dadurch erhofft man sich kulturelle Ähnlichkeit, ein besseres Verständnis aufgrund von politischen oder historischen Gemeinsamkeiten und bessere Bildung der Mitarbeiter des Outsourcing-Partners (Kvedaravičienė, 2008).

\subsection{Charakterisierung von kleinen und mittleren Unternehmen}

Nach den Empfehlung der EU sind Unternehmen mit weniger als 250 Mitarbeitern und einem Jahresumsatz von höchstens 50 Millionen Euro oder einer Jahresbilanzsumme von maximal 43 Millionen Euro der Unternehmensklasse der kleinen und mittleren Unternehmen zuzuordnen („Empfehlung der Kommission vom 6. Mai 2003 betreffend die Definition der Kleinstunternehmen sowie der kleinen und mittleren Unternehmen (Text von Bedeutung für den EWR) (Bekannt gegeben unter Aktenzeichen K(2003) 1422)“, 2003).

Damit sind die Möglichkeiten des Outsourcings stärker beschränkt als bei großen Unternehmen, weil personelle und finanzielle Mittel fehlen. Das betrifft insbesondere Ausgründungen (hoher personeller Aufwand) und Offshoring (hoher finanzieller Aufwand, insbesondere Suche und Kontrolle des Outsourcing-Partners).

\subsection{Chancen und Risiken des Outsourcings}

Eines der wichtigsten Argumente für Outsourcing sind Kostenargumente (Blattmann et al., 2019). Dazu zählt die Kostenreduktion durch die Lernkurve des Outsourcing-Partners und Skaleneffekten. Weiterhin werden die Kosten einer Dienstleistung transparent (Ortner, 2015) und bei entsprechender Vertragsgestaltung werden aus fixen Kosten für Gehälter variable Kosten für die in Anspruch genommene Leistung (Barthélemy, 2003).

Neben den Kosten der Leistungserstellung müssen auch die Kosten zur Suche von Outsourc\-ing-Partnern, Kosten der Vertragserstellung und -verhandlung, sowie Kosten zur Durchsetzung der vereinbarten Leistung mit betrachtet werden. Diese Kosten werden als Transaktionskosten bezeichnet. Die gemeinsame Betrachtung von Kosten der Leistungserstellung und Transaktionskosten ermöglicht eine Abwägung zwischen interner Leistungserstellung und Outsourcing (Coase, 1995). Es kann sein, dass der Kostenvorteil durch Outsourcing in Billiglohnländer durch hohe Transaktionskosten wie Informationskosten und Kosten der Durchsetzung verschwindet und das Outsourcing ein Verlustgeschäft wird. Durch das Internet sind insbesondere die Kosten der Informationsbeschaffung stark gesunken (Garicano \& Kaplan, 2001).

Bei hohen Investitionen auf Seite des Unternehmens, welches outsourcen möchte, in die Outsourcing-Beziehung steigt die Gefahr auf opportunistisches Verhalten des Outsourcing-Partners. Insbesondere steigt die Wahrscheinlichkeit, dass weniger Leistung oder schlechtere Qualität geliefert wird (Handley \& Benton, 2012). Ein Wechsel des Outsourcing-Partners oder eine Wiedereingliederung ins eigene Unternehmen ist aber meist mit hohen Kosten verbunden. Dass ein Wechsel nötig ist, kann entweder durch die schlechte Qualität bzw. geringere Leistung als vertraglich vereinbart notwendig sein (Hold-up Problem), oder eine hohe Investition in die Outsourcing-Beziehung kann zu einem Lock-in-Effekt führen (Ortner, 2015).

Neben geringeren Kosten ist vor allem eine Fokussierung auf Kernprozesse des eigenen Unternehmens durch Outsourcing möglich. Getreu dem Motto “Do what you do best, outsource the rest” führt eine Spezialisierung zu geringeren Kosten und damit zu höherem Gewinn durch eine Verbesserung der Wertschöpfungskette (Siems \& Ratner, 2003). Es ergeben sich Potentiale zur Qualitätssteigerung und das Risiko für Nicht-Kernprozesse wird auf den Outsourcing-Partner übertragen. Damit ist auch ein gewisser Know-How-Verlust einhergehend und Vertraulichkeit und Datenschutz werden umso wichtiger (Ortner, 2015).

Gerüchte, dass der eigene Arbeitsplatz outgesourct wird, können zu Demotivation und damit auch zu geringerer Produktivität führen, wenn Mitarbeitern keine zukünftige Perspektive im Unternehmen geboten werden kann. (Barthélemy, 2003; Ortner, 2015).

\subsection{Kritische Erfolgsfaktoren in der Literatur}

Laut Szczutkowski (o. J.) sind kritische Erfolgsfaktoren, Faktoren, die über den Erfolg einer Unternehmung entscheiden. Sind diese Faktoren erfüllt, so wird auch die Unternehmung erfolgreich sein, bei Abwesenheit dieser Faktoren ist der Erfolg der Unternehmung gefährdet.

Wie schon ausgeführt, sollten keine Kernprozesse outgesourct werden, sondern nur Hilfsprozesse. Dazu muss aber im Unternehmen dokumentiert sein, was Kern- und Hilfsprozesse sind. Zudem sollten diese standardisiert sein, um einfacher und schneller auf Probleme reagieren zu können und im Zweifel den Outsourcing-Partner einfacher und schneller wechseln zu können (Ortner, 2015). Embleton und Wright (1998) haben 5 Kriterien ermittelt, die helfen, Hilfsprozesse von Kernprozessen zu unterscheiden:
\begin{itemize}
	\item Ist der Prozess routiniert?
	\item Ist der Prozess standardisiert und klar umrissen?
	\item Kann der Erfolg des Prozesses einfach gemessen und gemanagt werden?
	\item Gibt es etablierte Anbieter für diesen Prozess?
	\item Sind diese Anbieter in einem Wettbewerbsmarkt?
\end{itemize}
Wenn diese Fragen bejaht werden können, ist sichergestellt, dass der Prozess zuverlässig zu einem günstigen Preis und hoher Servicequalität an den Outsourcing-Partner übergeben werden kann. Es bedarf also einer Machbarkeitsstudie, mit der man die ersten drei dieser fünf Fragen beantworten kann. Durch Marktanalyse und das Einholen von Angeboten verschiedener Anbieter lassen sich die letzten beiden Fragen beantworten (Ortner, 2015).

Hat man sich für einen Outsourcing-Partner entschieden, so geht es nun um die Vertragsgestaltung. Hier muss sowohl kurz- als auch langfristig gedacht werden und es müssen konkrete Ziele vereinbart werden. Zudem bietet es sich an, die gegenseitigen Erwartungshaltungen von Provider, Kunden, Mitarbeitern und Führungskräften zu klären, um spätere Enttäuschungen zu vermeiden (Embleton \& Wright, 1998; Ortner, 2015).

Nach Abschluss des Outsourcing-Vertrages muss die Beziehung zwischen Unternehmen und Provider konstant gepflegt und evaluiert werden, um Probleme frühzeitig zu erkennen. Es müssen Schnittstellen für den Informationsfluss geschaffen werden; dazu ist entsprechendes Know-How wichtig und es müssen Ressourcen dafür bereitgestellt werden. Im eigenen Unternehmen müssen die Führungskräfte voll hinter der Outsourcing-Entscheidung stehen und auch klar die zukünftige Rolle von Mitarbeitern des outgesourcten Bereiches mit diesen besprechen. Es gibt drei Möglichkeiten (Embleton \& Wright, 1998; Ortner, 2015):
\begin{itemize}
	\item Im Unternehmen bleiben: Es fallen neue Tätigkeiten im Unternehmen an, z.B. Kontrolle des Providers. Diese Mitarbeiter bringen schon das entsprechende Wissen mit, um die Qualität der Dienstleistung zu überprüfen. Diese neue Rolle und die sich daraus ergebenden neuen Anforderungen sollten mit den Mitarbeitern besprochen werden. Hier bietet sich auch die aktive Mitgestaltung des Mitarbeiters an seiner neuen Rolle an.
	\item Zum Outsourcing-Partner wechseln: Durch den neuen Auftrag braucht der Outsourc\-ing-Partner neue Mitarbeiter. Für den Übergang von Unternehmen zum Partner gibt es zwei Möglichkeiten: Zum einen den Schlussstrich-Ansatz, bei dem das Unternehmen weder Bedingungen stellt noch über den Job verhandelt. Dieser Ansatz ist deutlich zeit- und kostengünstiger als der schrittweise Übergang, bei dem das Unternehmen über den Job und die Bedingungen verhandelt. Dieser Ansatz ist nur dann vertretbar, wenn es besonders auf die Erfahrung und die Fähigkeiten des Mitarbeiters ankommt (Žiković \& Rupcic, 2004).
	\item Das Unternehmen verlassen: Das ist der denkbar schlechteste Fall. Die Mitarbeiter werden weder im Unternehmen noch im Outsourcing-Partner gebraucht und verlassen das Unternehmen. Das kann sich negativ auf die übrige Belegschaft auswirken.
\end{itemize}

Zudem muss die Stimmung unter den Mitarbeitern beobachtet werden. Die meisten Mitarbeiter sind Outsourcing logischerweise negativ aufgeschlossen, da ihr Arbeitsplatz in Gefahr sein könnte. Eine klare Kommunikation über Ziele, Erwartungen, neue Prioritäten oder Rollen ist daher zwingend notwendig, sodass die sozialen Kosten des Outsourcings minimiert werden. Sonst kann es passieren, dass sich Mitarbeiter von innen gegen das Unternehmen richten, nur um die Aufmerksamkeit des Managements zu erhalten (Embleton \& Wright, 1998).

\section{Methoden}

\section{Ergebnisse}



%\section{Grundlagen und Aufbau}
%\label{kap:einfuehrung}
%
%\subsection{Dokument-Parameter}
%    Als Grundlage der Arbeit muss einer der folgenden Parameter als Art der Arbeit gewählt werden: \textbf{foseminar}, \textbf{beleg}, \textbf{bachelor}, \textbf{master}, \textbf{diplom}. Der entsprechende Parameter ist am Anfang des Quellcodes (der \verb|.tex|-Datei) per \verb|\documentclass[...]{isih}| in den eckigen Kammern einzutragen. Weitere Parameter sind am Anfang des Quellcodes dieser Datei zu finden. Die Parameter der Arbeit umfassen persönliche Daten und Daten zu der Arbeit an sich. Diese sind ebenfalls am Anfang des Quellcodes in der (der \verb|.tex|-Datei) einzutragen. 
%    
%\subsection{Pre: Abstract, Auflistungen und Abkürzungsverzeichnis}
%    Ein \textbf{Abstract}, als Zusammenfassung der Arbeit, gehört an den Anfang. Hier sollte in einigen Sätzen der Kontext, das Problem, die Ziele, die Lösung und die Evaluierung erläutert werden. Dabei sind 250--350~Wörter üblich.
%    
%    Das \textbf{Inhaltsverzeichnis}, \textbf{Abbildungsverzeichnis}, \textbf{Tabellenverzeichnis}, \textbf{Quellcodeverzeichnis} und \textbf{Formelverzeichnis} werden automatisch aus dem Inhalt vom Haupt-Text generiert.
%
%    Ein \textbf{Abkürzungsverzeichnis} folgt direkt nach den anderen Auflistungen. Es muss manuell und konsistent gepflegt werden. Die Abkürzungen sind in alphabetischer Reihenfolge angegeben. Die erste Nennung im Haupt-Text ist (am besten kursiv) auszuschreiben und die Abkürzung in Klammern dahinter zu nennen, zB. \textit{Technische Universität Dresden} (TUD).
%
%
%\subsection{Post: Anhänge und Erklärung}
%    \textbf{Anhänge} sollten Dinge enthalten, die nicht notwendigerweise im Haupt-Text stehen müssen (Interviews, viele und große Grafiken, lange Tabellen, etc.) Dazu werden normale Kapitel mit \verb|\section{}| und \verb|\subsection{}| am Ende der Datei angelegt (nach dem \verb|\appendix| Keyword; siehe Anhang~\ref{anh:links}). Die Formatierung der Kapitel wird automatisch übernommen.
%    
%    Die \textbf{Ehrenwörtliche Erklärung} steht am Schluss und muss am Ende unterzeichnet eingereicht werden. Bei der Abgabe muss sowohl das PDF als auch der \LaTeX-Quellcode eingereicht werden. 
%
%\subsection{Änderungen am Template}
%    Das Template ist mit den Designregeln des ISIH-Lehrstuhls abgestimmt. Abweichungen sollten mit allen Beteiligten besprochen sein. Das Template gibt keine Garantie für Korrektheit und/oder Funktionalität. \textcolor{red}{Jeder Benutzer ist selbst für sein Dokument verantwortlich und hat es (unabhängig vom Template) so einzureichen, wie es abgesprochen und/oder erwartet wird.} 
%
%\section{Struktur und Überschriften}
%\label{kap:struktur}
%    Die Kapitel sind so zu strukturieren, dass die mit dem Inhalt der Arbeit stimmig sind. Weiterhin sollte jedes Kapitel mit einem \verb|\label{}| versehen sein, damit darauf innerhalb der Arbeit  (per \verb|\ref{}|) referenziert werden kann.
%    
%    Aus technischen Gründen ist für ein Kapitel der Befehl \verb|\section{}| (als oberste Struktur\-einheit) zu nutzen. Auf nicht nummerierte Kapitel sollte verzichtet werden.
%
%\subsection{Unterüberschriften}
%    Zur Unterteilung von Hauptunterschriften sind Unterüberschriften per \verb|\subsection{}| zu nutzen. Auch hier auf ein entsprechendes \verb|\label{}| achten.
%    
%    Ein Unterkapitel sollte nur dann genutzt werden, wenn nicht nur eines sondern mehrere davon innerhalb eines Hauptkapitels benutzt werden. Die relevanten Informationen stehen dabei immer nur in den ``Blättern'' der Kapitelstruktur: sollten Unterkapitel benutzt werden, ist im generischen (vorgelagerten) Hauptteil keine relevante Information vorhanden und es gibt mindestens ein weiteres Unterkapitel. Dies gilt für alle nachgelagerten, niederen Kapitelstrukturen ebenfalls.
%
%\subsubsection{Niedere Überschriften}
%    Niedere Kapitelüberschriften per \verb|\subsubsection{}| folgen den Bedingungen von normalen Unterkapiteln.
%
%\paragraph{Paragraphen}
%    Paragraphen (via \verb|\paragraph{}|) sind für kleinere Unterteilung innerhalb eines Textes (Absätze) gedacht. Paragraphen (und Unterparagraphen) erscheinen nicht mehr im Inhaltsverzeichnis.
%
%\subparagraph{Unterparagraphen}
%    Sollte es notwendig sein, können Paragraphen weiter unterteilt werden. Dazu ist \verb|\subparagraph{}| zu nutzen. Generell sollte von einer Unterteilung von Paragraphen in Unterparagraphen abgesehen werden.
%
%
%\section{Umgebungen}
%\label{kap:umgebungen}
%
%\subsection{Bilder}
%    Bilder werden per \verb|\begin{figure}|-Umgebung eingebunden. Die Abbildung \ref{fig:bild} zeigt ein entsprechendes Beispiel-Bild. Das Bild sollte normalerweise per \verb|[t]|-Parameter oben auf der Seite positioniert sein, notfalls ist aber auch unten (per \verb|[b]|) möglich. Ein \verb|\centering| sorgt für eine mittige Platzierung. Das \verb|\label{}| sollte auch hier gesetzt werden, damit man es per \verb|\ref{}| im Text referenzieren kann. Schlussendlich ergibt ein \verb|\caption[]{}| die Bildunterschrift: in eckigen Klammern kann eine Kurzvariante (für das Abbildungsverzeichnis) eingetragen sein (falls notwendig), und in geschweiften Klammen der normale Text unter dem Bild. Nummeriert wird das Bild automatisch. Weitere Infos und Links sind im Anhang~\ref{anh:links} zu finden.
%
%    \begin{figure}[t]
%        \centering
%        \includegraphics[width=11cm]{testbild.jpg}
%        \caption[Kurze Bild-Caption]
%                {Lange Bild-Caption, die nur im Dokument selbst (unter dem Bild) zu sehen ist}
%        \label{fig:bild}
%    \end{figure}
%
%
%\subsection{Tabellen}
%    Tabellen sind, ähnlich wie Bilder, in einer \verb|\begin{table}|-Umgebung zu setzen. Eine Beispiel-Tabelle findet sich in Tabelle~\ref{tab:uebersicht}. Auch Tabellen sind mit \verb|[t]| vorzugsweise oben zu platzieren. Es wird die \verb|\begin{tabularx}|-Umgebung für die innere Tabellenstruktur empfohlen, da hier der Seitenplatz variabel genutzt werden kann (ein \verb|X| in der Spaltendefinition). Es werden auf vertikale Begrenzungsrahmen verzichtet, die Kopfzeile ist mit \verb|\textbf{}| zu versehen. Horizontale Linien sind in der Kopfzeile und am Ende der Tabelle zu setzen, die Zeilen an sich per vorgelagerten \verb|\grayrow| abwechselnd farbig zu markieren (Start mit grauer Zeile). Ein \verb|\caption[]{}| gehört ebenfalls dazu. Weitere Infos zu Tabellen befinden sich im Anhang~\ref{anh:links}.
%
%    \begin{table}[t]
%    \begin{tabularx}{\textwidth}{l X X}
%        \hline
%        \textbf{Spalte 1} & \textbf{Spalte 2} & \textbf{Spalte 3} \\
%        \hline
%        \grayrow
%        Inhalt            & Inhalt            & Inhalt \\
%        Inhalt            & Inhalt            & Inhalt \\
%        \grayrow
%        Inhalt            & Inhalt            & Inhalt \\
%        \hline
%    \end{tabularx}
%    \caption[Kurze Tabellen-Caption]{Eine Tabelle mit drei Spalten und drei Zeilen}
%    \label{tab:uebersicht}
%    \end{table}
%
%
%\subsection{Auflistungen und Beschreibungen}
%    Die gängigen und vordefinierten Auflistungen \verb|itemize|, \verb|enumerate| sowie \verb|description| werden empfohlen. 
%
%\subsection{Mathematische Formeln}
%    Für kurze mathematische inline-Formeln kann die \verb|$...$|-Notation genutzt werden, wie zum Beispiel $E = mc^2$. Für alle Formeln, die für den Inhalt essenziell sind, referenziert werden und/oder länger ausfallen, sollte die \verb|\begin{equation}|-Umgebung genutzt werden:
%    
%    
%    \begin{equation}\label{math:1}
%        y_b = \sum_{i=0}^{n} a \times x^i - \left( \frac{a^2}{2\sqrt{b}} \right)
%    \end{equation}
%    \equationlist{Beispiel-Formel mit Verzeichniseintrag}
%    
%    Eine Referenz auf die entsprechende Formel ist dann per \verb|\ref{}| möglich, wie etwa ``siehe Gleichung~\ref{math:1}''. Dafür muss jede Formeln in einer eigenen Umgebung platziert werden. Unterhalb der Umgebung ist der Befehl \verb|\equationlist{}| zu nutzen, damit ein Eintrag im Formelverzeichnis erfolgt. Falls nicht notwendig, kann das Formelverzeichnis (\verb|\listofusedequations|) entfernt werden (nach Absprache mit dem Beteuer). Weitere Informationen finden sich unter Anhang~\ref{anh:links}.
%
%\subsection{Quellcode}
%    Für Quellcode ist in den meisten Fällen eine \verb|\begin{verbatim}|-Umgebung ausreichend. Bei Quellcode, welcher erläutert wird und wesentlich zum Inhalt beiträgt, kann eine Benutzung von \verb|\begin{lstlisting}|-Umgebung in sinnvoll sein (siehe Quellcode~\ref{code1}). Die Formatierung sollte mit dem Betreuer abgesprochen sein, da hier viele Optionen möglich sind. Zeilennummern sollten immer mit eingefügt werden. Das \verb|label| und \verb|caption| sind in der Definition zu setzen. Ein \verb|float| setzt den Quellcode nach oben auf die Seite. Sowohl die \verb|listings|-Definition als auch der Quellcode kann ausgelagert werden. Nach Absprache kann das Quellcodeverzeichnis (\verb|\lstlistoflistings|) entfernt werden. Weitere Infos befinden sind im Anhang~\ref{anh:links}.
%    
%    \begin{lstlisting}[
%        language=C++,
%        label={code1},
%        caption={[Kurze Code-Caption zu C++] Beispielfunktion in \texttt{HelloWorld.cpp} mit einer langen Caption},
%        float,
%    ]
%#include <iostream>
%using namespace std;
%int main(){
%    cout << "Hello, World!";
%    return 0;
%}
%    \end{lstlisting}
%
%
%\section{Referenzen und Zitate}
%
%\subsection{Wissenschaftliche Quellen}
%Grundsätzlich sind alle (Gedanken-)Ansätze, Methoden, Aussagen und Ergebnisse, die nicht vom Studierenden selbst stammen, mit Quellen zu belegen. Die Professur empfiehlt unbedingt die Nutzung eines Literaturverwaltungsprogramms (bspw. \href{https://www.zotero.org/}{Zotero}, \href{https://www.mendeley.com}{Mendeley}, \href{https://www.citavi.com/de}{Citavi}). Dies reduziert erheblich den Arbeitsaufwand beim wissenschaftlichen Zitieren und bei der Erstellung des Literaturverzeichnisses.
%
%Wikipedia und sonstige Online-Enzyklopädien sind keine wissenschaftliche Quellen! Jedoch kann Wikipedia etc. für die Suche nach weiterführende Literatur hilfreich sein. Nur in Absprache mit dem Betreuer darf Wikipedia (zB. für einfache Begriffserklärungen) benutzt werden. 
%
%\subsection{Zitier-Stil und Vorgaben}
%    Im Fließtext erfolgt die Referenzierung durch Nutzung des Zitierstils der (kursiv) American Psychological Association, kurz \textit{APA}. Die entsprechenden Definitionen sind bereits im Dokument eingepflegt. Zum Zitieren einer Quelle wird die \verb|natbib|-Bibliothek benutzt, welche viele verschiedene Möglichkeiten der Nennung anbietet. Wie entsprechende Literatur korrekt in diesem Stil zitiert wird, kann folgender Seite entnommen werden:
%    \begin{center}
%        \href{https://irsc.libguides.com/apa/referenceexamples}{\texttt{https://irsc.libguides.com/apa/referenceexamples}}
%    \end{center}
%    Im Anhang~\ref{anh:links} befinden sich weitere Informationen zu verschiedenen Styles und die entsprechenden \verb|natbib|-Befehle. Die am häufigsten im Text genutzten Zitierweisen sind folgende:
%    \begin{itemize}
%        \item erstmalige Nennung mit allen Autoren \verb|\citep|*\verb|{}|: \citep*{book}
%        \item simple Nennung \verb|\citep{}|: \citep{book}
%        \item mit Prefix und Suffix \verb|\citep[][]{}|: \citep[siehe][S.~740\,ff.]{proc}
%        \item mehrere Angaben \verb|\citep{ , }|: \citep{article,book}
%        \item aktive Nennung \verb|\citet{}|: ``wie von \citet{proc2} aufgezeigt''
%    \end{itemize}
%    
%    \textit{Bitte fügen Sie immer die Seitenzahl als Suffix an jede Quelle hinzu, wenn Sie auf eine bestimmte Stelle im Dokument verweisen!} Damit die Referenzierung der Literatur ordentlich funktioniert, muss eine \verb|.bib|-Datei mit den Quellen im Projekt existieren und eingebunden werden (in diesem Template heißt diese \verb|library.bib|). Nach Möglichkeit ist die referenzierte Literatur komplett mit allen Metadaten einzutragen. Es kann hierbei eine Literaturverwaltung benutzt werden, welche automatisch die benötige \verb|.bib|-Datei erzeugt.
%    
%
%\subsection{Fußnoten, Links, Websites}
%    Die Benutzung von \textbf{Fußnoten} sollte vermieden werden. Ansonsten sind Fußnoten nur zur weiteren Erklärung da, welche den Lesefluss stören würden, aber niemals für Literaturangaben. Davon ausgenommen sind Verweise auf Websiten\footnote{Wie zum Beispiel Verweise auf News-Websites, Code-Repositories oder ähnliche, nicht als Quellen benutzte URLs.}. Dazu muss der Befehl \verb|\footnote{}| benutzt\footnote{Die Nummerierung geschieht hierbei automatisch.} werden.
%
%    Für \textbf{Dokument-interne Referenzen} wird der Befehl \verb|\ref{}| benutzt. Generell sollten Vorwärts\-referenzen eher selten benutzt werden, besser sind Rückwärts\-referenzen im Dokument.
%
%    Für \textbf{Weblinks} steht der Befehl \verb|\href{}{}| zur Verfügung (vorn der Link, hinten der Text). Web-Referenzen sollten im Allgemeinen nicht im Text sondern als Fußnote oder Quelle genutzt werden.
%
%
%
%

%   Ende des Hauptteils
%%%%%%%%%%%%%%%%%%%%%%%%%%%%%%%%%%%%%%%%%%%%%%%%%%%%%%%%%%%%%%%%%%%%%%%%

\bibliographystyle{apacite}
\bibliography{library}
\begin{closing}
\appendix

%%%%%%%%%%%%%%%%%%%%%%%%%%%%%%%%%%%%%%%%%%%%%%%%%%%%%%%%%%%%%%%%%%%%%%%%
%   Beginn des Anhangs


%   Ende des Anhangs
%%%%%%%%%%%%%%%%%%%%%%%%%%%%%%%%%%%%%%%%%%%%%%%%%%%%%%%%%%%%%%%%%%%%%%%%

\declaration
\end{closing}
\end{document}