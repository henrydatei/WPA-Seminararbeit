\subsection{Definition von Outsourcing}

Outsourcing beschreibt die Auslagerung von Unternehmensprozessen in andere Unternehmen. Diese Prozesse könnten auch unternehmensintern vorhanden sein, aber durch effektivere und effizientere Bearbeitung des Prozesses durch ein anderes Unternehmen kann ein Wettbewerbsvorteil entstehen. Es geht dabei nicht nur um das Kaufen von Produkten bzw. Dienstleistungen, sondern auch um das Abgeben von Verantwortung und Know-how an den Zulieferer (McCarthy \& Anagnostou, 2004).

Wie in \cref{dimensionen} dargestellt, gibt es verschiedene Dimensionen des Outsourcings: Unternehmenszugehörigkeit, Leistungsumfang, Zeit, Anzahl der Leistungsersteller und Entfernung zwischen den Outsourcing-Partnern.

\begin{figure}[h]
	\centering
	\includegraphics[width=0.75\textwidth]{./pics/testbild.jpg}
	\caption[Dimensionen des Outsourcings]{Dimensionen des Outsourcings, (Hermes \& Schwarz, 2005)}
	\label{dimensionen}
\end{figure}

Die Dimension Unternehmenszugehörigkeit hat die Ausprägungen externes und internes Outsourcing. Externes Outsourcing meint Outsourcing an Fremdfirmen außerhalb eines Konzerns; im Gegensatz dazu kann internes Outsourcing zum Beispiel das Outsourcing an eine andere Abteilung oder an eine Ausgründung sein. Beim internen Outsourcing macht sich ein Unternehmen nicht von Fremdfirmen abhängig und behält sein Know-how (Hermes \& Schwarz, 2005).

Das Auslagern ganzer Prozesse wie zum Beispiel Buchführung, Logistik oder Einkauf wird als Business Process Outsourcing bezeichnet und ist insbesondere dann sinnvoll, wenn sich einzelne Funktionen des Unternehmens nicht leicht isolieren lassen (Dittrich \& Braun, 2004). Beim partiellen Outsourcing werden nur einzelne Aufgaben, zum Beispiel Übersetzung von Texten, an den Outsourcing-Partner übergeben. Der Outsourcing-Partner kennt den unternehmensinternen Zusammenhang aber nicht. Wenn sich Prozesse im Unternehmen gut abgrenzen lassen, wie zum Beispiel ein Sicherheitsdienst, kann ein totales Outsourcing sinnvoll sein. Die Anforderungen lassen sich hierbei gut definieren und messen (Anderson, 1997).

Wird eine Outsourcing-Entscheidung wieder rückgängig gemacht, spricht man von Backsourcing. IT-Dienstleistungen werden häufig aufgrund ihrer hohen Kosten an Fremdfirmen abgegeben, aber wegen schlechter Qualität wieder ins eigene Unternehmen zurückgeholt (von Bary et al., 2020). Beim Insourcing evaluiert man die Outsourcing-Option, versucht aber das gleiche Ergebnis des Outsourcings mit internen Mitteln zu erreichen (Hirschheim \& Lacity, 2000).

Gibt es für ein Thema oder Aufgabenbereich nur einen Outsourcing-Partner, so handelt es sich um Single-Sourcing; bei mehreren Partnern um Multi-Sourcing. Beim Single-Sourcing läuft man als Unternehmen Gefahr, dass man sich in einen zu große Abhängigkeit des Partners begibt. Multi-Sourcing löst dieses Problem, hat aber einen Preis: mehrere Transaktionen erhöhen die Transaktionskosten. Zudem ist ein gutes Management der Outsourcing-Beziehungen notwendig (Hermes \& Schwarz, 2005).

Beim Offshoring sitzt der Outsourcing-Partner in Ländern mit deutlich geringen Lohnkosten; häufigstes Beispiel ist hierfür Indien (Bongartz, 2003; Farrell, 2004). Beim Nearshoring sitzt der Outsourcing-Partner geografisch näher am eigenen Unternehmen, das heißt es werden nur Unternehmens- und Landesgrenzen überquert, aber keine Zeitzonen (Zieris \& Salinger, 2013), Dadurch erhofft man sich kulturelle Ähnlichkeit, ein besseres Verständnis aufgrund von politischen oder historischen Gemeinsamkeiten und bessere Bildung der Mitarbeiter des Outsourcing-Partners (Kvedaravičienė, 2008).

\subsection{Charakterisierung von kleinen und mittleren Unternehmen}

Nach den Empfehlung der EU sind Unternehmen mit weniger als 250 Mitarbeitern und einem Jahresumsatz von höchstens 50 Millionen Euro oder einer Jahresbilanzsumme von maximal 43 Millionen Euro der Unternehmensklasse der kleinen und mittleren Unternehmen zuzuordnen („Empfehlung der Kommission vom 6. Mai 2003 betreffend die Definition der Kleinstunternehmen sowie der kleinen und mittleren Unternehmen (Text von Bedeutung für den EWR) (Bekannt gegeben unter Aktenzeichen K(2003) 1422)“, 2003).

Damit sind die Möglichkeiten des Outsourcings stärker beschränkt als bei großen Unternehmen, weil personelle und finanzielle Mittel fehlen. Das betrifft insbesondere Ausgründungen (hoher personeller Aufwand) und Offshoring (hoher finanzieller Aufwand, insbesondere Suche und Kontrolle des Outsourcing-Partners).

\subsection{Chancen und Risiken des Outsourcings}

Eines der wichtigsten Argumente für Outsourcing sind Kostenargumente (Blattmann et al., 2019). Dazu zählt die Kostenreduktion durch die Lernkurve des Outsourcing-Partners und Skaleneffekten. Weiterhin werden die Kosten einer Dienstleistung transparent (Ortner, 2015) und bei entsprechender Vertragsgestaltung werden aus fixen Kosten für Gehälter variable Kosten für die in Anspruch genommene Leistung (Barthélemy, 2003).

Neben den Kosten der Leistungserstellung müssen auch die Kosten zur Suche von Outsourc\-ing-Partnern, Kosten der Vertragserstellung und -verhandlung, sowie Kosten zur Durchsetzung der vereinbarten Leistung mit betrachtet werden. Diese Kosten werden als Transaktionskosten bezeichnet. Die gemeinsame Betrachtung von Kosten der Leistungserstellung und Transaktionskosten ermöglicht eine Abwägung zwischen interner Leistungserstellung und Outsourcing (Coase, 1995). Es kann sein, dass der Kostenvorteil durch Outsourcing in Billiglohnländer durch hohe Transaktionskosten wie Informationskosten und Kosten der Durchsetzung verschwindet und das Outsourcing ein Verlustgeschäft wird. Durch das Internet sind insbesondere die Kosten der Informationsbeschaffung stark gesunken (Garicano \& Kaplan, 2001).

Bei hohen Investitionen auf Seite des Unternehmens, welches outsourcen möchte, in die Outsourcing-Beziehung steigt die Gefahr auf opportunistisches Verhalten des Outsourcing-Partners. Insbesondere steigt die Wahrscheinlichkeit, dass weniger Leistung oder schlechtere Qualität geliefert wird (Handley \& Benton, 2012). Ein Wechsel des Outsourcing-Partners oder eine Wiedereingliederung ins eigene Unternehmen ist aber meist mit hohen Kosten verbunden. Dass ein Wechsel nötig ist, kann entweder durch die schlechte Qualität bzw. geringere Leistung als vertraglich vereinbart notwendig sein (Hold-up Problem), oder eine hohe Investition in die Outsourcing-Beziehung kann zu einem Lock-in-Effekt führen (Ortner, 2015).

Neben geringeren Kosten ist vor allem eine Fokussierung auf Kernprozesse des eigenen Unternehmens durch Outsourcing möglich. Getreu dem Motto “Do what you do best, outsource the rest” führt eine Spezialisierung zu geringeren Kosten und damit zu höherem Gewinn durch eine Verbesserung der Wertschöpfungskette (Siems \& Ratner, 2003). Es ergeben sich Potentiale zur Qualitätssteigerung und das Risiko für Nicht-Kernprozesse wird auf den Outsourcing-Partner übertragen. Damit ist auch ein gewisser Know-How-Verlust einhergehend und Vertraulichkeit und Datenschutz werden umso wichtiger (Ortner, 2015).

Gerüchte, dass der eigene Arbeitsplatz outgesourct wird, können zu Demotivation und damit auch zu geringerer Produktivität führen, wenn Mitarbeitern keine zukünftige Perspektive im Unternehmen geboten werden kann. (Barthélemy, 2003; Ortner, 2015).

\subsection{Kritische Erfolgsfaktoren in der Literatur}

Laut Szczutkowski (o. J.) sind kritische Erfolgsfaktoren, Faktoren, die über den Erfolg einer Unternehmung entscheiden. Sind diese Faktoren erfüllt, so wird auch die Unternehmung erfolgreich sein, bei Abwesenheit dieser Faktoren ist der Erfolg der Unternehmung gefährdet.

Wie schon ausgeführt, sollten keine Kernprozesse outgesourct werden, sondern nur Hilfsprozesse. Dazu muss aber im Unternehmen dokumentiert sein, was Kern- und Hilfsprozesse sind. Zudem sollten diese standardisiert sein, um einfacher und schneller auf Probleme reagieren zu können und im Zweifel den Outsourcing-Partner einfacher und schneller wechseln zu können (Ortner, 2015). Embleton und Wright (1998) haben 5 Kriterien ermittelt, die helfen, Hilfsprozesse von Kernprozessen zu unterscheiden:
\begin{itemize}
	\item Ist der Prozess routiniert?
	\item Ist der Prozess standardisiert und klar umrissen?
	\item Kann der Erfolg des Prozesses einfach gemessen und gemanagt werden?
	\item Gibt es etablierte Anbieter für diesen Prozess?
	\item Sind diese Anbieter in einem Wettbewerbsmarkt?
\end{itemize}
Wenn diese Fragen bejaht werden können, ist sichergestellt, dass der Prozess zuverlässig zu einem günstigen Preis und hoher Servicequalität an den Outsourcing-Partner übergeben werden kann. Es bedarf also einer Machbarkeitsstudie, mit der man die ersten drei dieser fünf Fragen beantworten kann. Durch Marktanalyse und das Einholen von Angeboten verschiedener Anbieter lassen sich die letzten beiden Fragen beantworten (Ortner, 2015).

Hat man sich für einen Outsourcing-Partner entschieden, so geht es nun um die Vertragsgestaltung. Hier muss sowohl kurz- als auch langfristig gedacht werden und es müssen konkrete Ziele vereinbart werden. Zudem bietet es sich an, die gegenseitigen Erwartungshaltungen von Provider, Kunden, Mitarbeitern und Führungskräften zu klären, um spätere Enttäuschungen zu vermeiden (Embleton \& Wright, 1998; Ortner, 2015).

Nach Abschluss des Outsourcing-Vertrages muss die Beziehung zwischen Unternehmen und Provider konstant gepflegt und evaluiert werden, um Probleme frühzeitig zu erkennen. Es müssen Schnittstellen für den Informationsfluss geschaffen werden; dazu ist entsprechendes Know-How wichtig und es müssen Ressourcen dafür bereitgestellt werden. Im eigenen Unternehmen müssen die Führungskräfte voll hinter der Outsourcing-Entscheidung stehen und auch klar die zukünftige Rolle von Mitarbeitern des outgesourcten Bereiches mit diesen besprechen. Es gibt drei Möglichkeiten (Embleton \& Wright, 1998; Ortner, 2015):
\begin{itemize}
	\item Im Unternehmen bleiben: Es fallen neue Tätigkeiten im Unternehmen an, z.B. Kontrolle des Providers. Diese Mitarbeiter bringen schon das entsprechende Wissen mit, um die Qualität der Dienstleistung zu überprüfen. Diese neue Rolle und die sich daraus ergebenden neuen Anforderungen sollten mit den Mitarbeitern besprochen werden. Hier bietet sich auch die aktive Mitgestaltung des Mitarbeiters an seiner neuen Rolle an.
	\item Zum Outsourcing-Partner wechseln: Durch den neuen Auftrag braucht der Outsourc\-ing-Partner neue Mitarbeiter. Für den Übergang von Unternehmen zum Partner gibt es zwei Möglichkeiten: Zum einen den Schlussstrich-Ansatz, bei dem das Unternehmen weder Bedingungen stellt noch über den Job verhandelt. Dieser Ansatz ist deutlich zeit- und kostengünstiger als der schrittweise Übergang, bei dem das Unternehmen über den Job und die Bedingungen verhandelt. Dieser Ansatz ist nur dann vertretbar, wenn es besonders auf die Erfahrung und die Fähigkeiten des Mitarbeiters ankommt (Žiković \& Rupcic, 2004).
	\item Das Unternehmen verlassen: Das ist der denkbar schlechteste Fall. Die Mitarbeiter werden weder im Unternehmen noch im Outsourcing-Partner gebraucht und verlassen das Unternehmen. Das kann sich negativ auf die übrige Belegschaft auswirken.
\end{itemize}

Zudem muss die Stimmung unter den Mitarbeitern beobachtet werden. Die meisten Mitarbeiter sind Outsourcing logischerweise negativ aufgeschlossen, da ihr Arbeitsplatz in Gefahr sein könnte. Eine klare Kommunikation über Ziele, Erwartungen, neue Prioritäten oder Rollen ist daher zwingend notwendig, sodass die sozialen Kosten des Outsourcings minimiert werden. Sonst kann es passieren, dass sich Mitarbeiter von innen gegen das Unternehmen richten, nur um die Aufmerksamkeit des Managements zu erhalten (Embleton \& Wright, 1998).