Der globale Outsourcing-Markt wächst beständig. Das Marktforschungsinstitut ISG ermittelt jedes Quartal das kumulierte Volumen aller Outsourcing-Abschlüsse der Privatwirtschaft und öffentlichen Verwaltung von mindestens 4 Millionen Euro. Das jährliche Volumen der Abschlüsse der Region EMEA (Europa, naher Osten und Afrika) stieg dabei von 2014 bis 2019 von 12 Milliarden auf 19,2 Milliarden US-Dollar - ein Anstieg von 60 \% (siehe \cref{auftragsvolumen}).

\begin{figure}[h]
	\centering
	\includegraphics[width=0.75\textwidth]{./pics/testbild.jpg}
	\caption[Auftragsvolumen von Outsourcing-Verträgen]{Jährliches Auftragsvolumen von Outsourcing-Verträgen in Milliarden US-Dollar, Daten aus (Jain \& Sachs, 2020; King, 2017; Matthews, 2019), eigene Berechnungen}
	\label{auftragsvolumen}
\end{figure}

Das Outsourcing verspricht unter anderem Kostenreduktion und -transparenz, sowie einfachere Skalierbarkeit von Leistungen, aber verbunden mit einem hohen Risko: Hat man als Unternehmen erstmals einen Anbieter gefunden, so sind mit einem Anbieterwechsel hohe Kosten und großer Zeitaufwand verbunden (Ortner, 2015). Es kommt also auf die richtige Wahl des Outsourcing-Anbieters an. In den letzten 20 Jahren wurde daher viel Forschung betrieben, um die kritischen Erfolgsfaktoren des Outsourcings zu ermitteln und so so das Risiko einer schlechten Outsourcing-Entscheidung zu minimieren.

Die Ergebnisse dieser Forschung stammen aber häufig aus Untersuchungen von Outsourcing-Verträgen großer Firmen (Barthélemy, 2003), obwohl insbesondere Unternehmen mit wenigen Mitarbeitern, also kleine und mittlere Unternehmen, auf Outsourcing von zum Beispiel IT oder des Rechnungswesens angewiesen sind und dies auch tun (Embleton \& Wright, 1998) (siehe \cref{mitarbeiter}).

\begin{figure}[h]
	\centering
	\includegraphics[width=0.75\textwidth]{./pics/testbild.jpg}
	\caption[Mitarbeiteranzahl von Unternehmen, die Outsourcing nutzen]{Mitarbeiteranzahl von Unternehmen, die Outsourcing nutzen, aus (Embleton \& Wright, 1998)}
	\label{mitarbeiter}
\end{figure}

Es stellt sich also folgende Forschungsfrage: \textit{Sind die in der Forschung ermittelten kritischen Erfolgsfaktoren für Outsourcing auch für kleine und mittlere Unternehmen gültig?} Dazu sollen anhand einer Fallstudie die Outsourcing-Beziehungen eines Unternehmens auf Erfolg oder Misserfolg untersucht werden. Dabei soll überprüft werden, ob sich deren Ergebnis (Erfolg oder Misserfolg) auf die in einer Literaturrecherche ermittelten Erfolgsfaktoren zurückführen lässt. Auf diese Weise leistet die Arbeit einen Beitrag zur Anwendbarkeit von Erfolgsfaktoren, die für Großunternehmen ermittelt wurde, auf kleine und mittlere Unternehmen.